%%% LaTeX-Vorlage Version 1.8 %%%

% Grundlegende Dokumenteneigenschaften gemäß DHBW-Vorgaben
\documentclass[a4paper,fontsize=11pt,oneside,parskip=half,headings=normal]{scrreprt} 
% \usepackage{showframe} % nur für Kontrolle der Ränder 

%%% Präambel einbinden (mit Festlegungen gemäß DHBW-Vorgaben) %%%
\input{template/_dhbw_praeambel.tex}

%%% Name der eigenen Literatur-Datenbank (ggf. anpassen) %%%
% \bibliography{includes/name_of_your_bibliography.bib}

\begin{document}
%%% Deckblatt einbinden %%% 
% Anpassungen nötig (Name, Titel etc.)
\input{includes/deckblatt.tex}

%%% Umstellung der Seiten-Nummerierung auf i, ii, iii ... %%%
\pagenumbering{Roman}

%%% Inhalts-, Abbildungs-, Tabellenverzeichnisse %%%
% sollen einzeilig gesetzt werden, um Platz zu sparen 
\begin{spacing}{1}
  \tableofcontents
  \clearpage
  \chapter*{Abkürzungsverzeichnis}
\addcontentsline{toc}{chapter}{Abkürzungsverzeichnis}

\begin{acronym}[DHBW]
  % Argument definiert die Breite der ersten Spalte anhand des längsten vorkommenden Eintrags
  \acro{CRM}{Customer Relationship Management}
  \acro{DHBW}{Duale Hochschule Baden-Württemberg}
  \acro{IEEE}{Institute of Electrical and Electronics Engineers}
  \acro{ITIL}{IT Infrastructure Library}
  \acro{RoI}{Return-On-Invest}
  \acro{UCS}{Universal Character Set}
  \acro{UTF-8}{8-Bit UCS Transformation Format}
\end{acronym}

\vspace{2em}


  \clearpage
  \thispagestyle{kapitelkopfzeile}
  \listoffigures
  \phantomsection
  \addcontentsline{toc}{chapter}{Abbildungsverzeichnis} % Abb.verz. ins Inh.verz. aufnehmen

  \clearpage
  \listoftables
  \phantomsection
  \addcontentsline{toc}{chapter}{Tabellenverzeichnis}   % Tab.verz. ins Inh.verz. aufnehmen
  \clearpage
  \lstlistoflistings
  \addcontentsline{toc}{chapter}{Listingsverzeichnis}   % Lst.verz. ins Inh.verz. aufnehmen
\end{spacing}

%%% Umstellung der Seiten-Nummerierung auf 1, 2, 3 ... %%%
\cleardoublepage
\pagenumbering{arabic}

%%% Ihr eigentlicher Inhalt %%%
% Empfehlung: strukturieren Sie Ihren Text in einzelnen Dateien 
% und binden Sie diese hier mit \input{includes/dateiname.tex} ein

%%% Ende des eigentlichen Inhalts %%%

% chapter  (end)

%%% Quellenverzeichnisse (keine Anpassung nötig) %%%
\clearpage
\literaturverzeichnis
%%% Ende Quellenverzeichnisse %%%


%%% Erklärung (keine Anpassungen nötig) %%%
% steht ganz am Ende des Dokuments
\cleardoublepage
\input{template/_dhbw_erklaerung.tex}
\end{document}
